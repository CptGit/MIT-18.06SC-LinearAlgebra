\PassOptionsToPackage{unicode=true}{hyperref} % options for packages loaded elsewhere
\PassOptionsToPackage{hyphens}{url}
%
\documentclass[]{article}
\usepackage{lmodern}
\usepackage{amssymb,amsmath}
\usepackage{ifxetex,ifluatex}
\usepackage{fixltx2e} % provides \textsubscript
\ifnum 0\ifxetex 1\fi\ifluatex 1\fi=0 % if pdftex
  \usepackage[T1]{fontenc}
  \usepackage[utf8]{inputenc}
  \usepackage{textcomp} % provides euro and other symbols
\else % if luatex or xelatex
  \usepackage{unicode-math}
  \defaultfontfeatures{Ligatures=TeX,Scale=MatchLowercase}
\fi
% use upquote if available, for straight quotes in verbatim environments
\IfFileExists{upquote.sty}{\usepackage{upquote}}{}
% use microtype if available
\IfFileExists{microtype.sty}{%
\usepackage[]{microtype}
\UseMicrotypeSet[protrusion]{basicmath} % disable protrusion for tt fonts
}{}
\IfFileExists{parskip.sty}{%
\usepackage{parskip}
}{% else
\setlength{\parindent}{0pt}
\setlength{\parskip}{6pt plus 2pt minus 1pt}
}
\usepackage{hyperref}
\hypersetup{
            pdfborder={0 0 0},
            breaklinks=true}
\urlstyle{same}  % don't use monospace font for urls
\setlength{\emergencystretch}{3em}  % prevent overfull lines
\providecommand{\tightlist}{%
  \setlength{\itemsep}{0pt}\setlength{\parskip}{0pt}}
\setcounter{secnumdepth}{0}
% Redefines (sub)paragraphs to behave more like sections
\ifx\paragraph\undefined\else
\let\oldparagraph\paragraph
\renewcommand{\paragraph}[1]{\oldparagraph{#1}\mbox{}}
\fi
\ifx\subparagraph\undefined\else
\let\oldsubparagraph\subparagraph
\renewcommand{\subparagraph}[1]{\oldsubparagraph{#1}\mbox{}}
\fi

% set default figure placement to htbp
\makeatletter
\def\fps@figure{htbp}
\makeatother


\date{}

\begin{document}

\hypertarget{unit-i-ax-b-and-the-four-subspaces}{%
\section{Unit I: Ax = b and the Four
Subspaces}\label{unit-i-ax-b-and-the-four-subspaces}}

\hypertarget{seesion-1.1-the-geometry-of-linear-equations}{%
\subsection{Seesion 1.1: The Geometry of Linear
Equations}\label{seesion-1.1-the-geometry-of-linear-equations}}

We have a system of equations: \[
\begin{aligned}
2x - y &= 0 \\
-x + 2y &= 3
\end{aligned}
\]

\hypertarget{row-picture}{%
\subsubsection{Row Picture}\label{row-picture}}

Line \(2x - y = 0\) and line \(-x + 2y = 0\) intersects at the point
\((1, 2)\), so \((1, 2)\) is the solution of the system of equations.
\textgreater{} Maybe I should draw a X-Y coordinates here\ldots{}

\hypertarget{column-picture}{%
\subsubsection{Column Picture}\label{column-picture}}

We rewrite the system of linear equations as a single equation:

\[
x\begin{bmatrix}2 \\-1\end{bmatrix} + y\begin{bmatrix}-1 \\2\end{bmatrix} = \begin{bmatrix}0 \\3\end{bmatrix}
\]

We see \(x\) and \(y\) as coefficients of column vectors:
\(\boldsymbol{v_1} = \begin{bmatrix}2 \\ -1\end{bmatrix}\) and
\(\boldsymbol{v_2} = \begin{bmatrix}-1 \\ 2\end{bmatrix}\), and the sum
\(x\boldsymbol{v_1} + y\boldsymbol{v_2}\) is called a \emph{linear
combination} of \(\boldsymbol{v_1}\) and \(\boldsymbol{v_2}\).

Geometrically, we can find one copy of \(\boldsymbol{v_1}\) added to two
copies of \(\boldsymbol{v_2}\) just equals the vector
\(\begin{bmatrix}0 \\3\end{bmatrix}\). Then the solution should be
\(x = 1, y =2\). \textgreater{} I will add a figure when time is
available \textgreater{}\_\textgreater{}

\hypertarget{matrx-picture}{%
\subsubsection{Matrx Picture}\label{matrx-picture}}

We rewrite the equations in our example as a compact form,

\[
A\boldsymbol{x} = \boldsymbol{b},
\]

that is

\[
\begin{bmatrix}2 & -1 \\ -1 & 2\end{bmatrix}\begin{bmatrix}x \\y\end{bmatrix} = \begin{bmatrix}0 \\3\end{bmatrix}
\]

\hypertarget{matrix-multiplication}{%
\subsubsection{Matrix Multiplication}\label{matrix-multiplication}}

\[
\begin{aligned}
\begin{bmatrix}2 & -1 \\ -1 & 2\end{bmatrix} \begin{bmatrix}1 \\ 2\end{bmatrix} = 1\begin{bmatrix}2 \\-1\end{bmatrix} + 2\begin{bmatrix}-1 \\2\end{bmatrix} = \begin{bmatrix}0 \\3\end{bmatrix}
\end{aligned}
\]

A matrix times by a vector is just \textbf{a linear combination of the
column vectors of the matrix}.

\end{document}
